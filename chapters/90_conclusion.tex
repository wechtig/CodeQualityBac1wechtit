%%%%%%%%%%%%%%%%%%%%%%%%%%%%%%%%%%%%%%%%%%%%%%%%%%%%%%%%%%%%%%%%%%%%%%%%%%%%%
%\chapter{Conclusion and Outlook}
\chapter{Fazit und Ausblick}
\label{chap:conclusion}
%%%%%%%%%%%%%%%%%%%%%%%%%%%%%%%%%%%%%%%%%%%%%%%%%%%%%%%%%%%%%%%%%%%%%%%%%%%%%
\chapterstart
Die Ergebnisse der Implementierung, die Evaluierung anhand der Testpersonen und die Vergleiche zu herkömmlichen Lösungen zeigen, dass die Ergebnisse der Statischen Code Analyse auf verschiedene Weise aufbereitet und angezeigt werden können, um den Code der Entwicklerinnen und Entwickler nachhaltig zu verbessern. Der Fokus hierbei liegt auf der Code Qualität, ein wichtiger Teil in der Softwareentwicklung.

Die Tools der Statischen Code Analyse bieten noch einige Möglichkeiten, die Daten weiter zu analysieren und aufzubereiten. Eine Desktop-Applikation oder eine Extension für eine Entwicklungsumgebung geben noch weitere Möglichkeiten vor. Ebenso kann eine Integration mit der Versionsverwaltung weitere Unterstützungsmöglichkeiten für Entwicklerteams liefern.

In dieser Arbeit wurde die Applikation unabhängig der verschiedenen Probleme und Faktoren entwickelt, die die Statische Code Analyse analysiert und als Fehler oder CodeSmell markiert. Fokussierungen in der Applikation auf einzelne Faktoren wie Security oder Abhängigkeiten können weitere Charts und Flow-Diagramme erfordern.

Weitere Untersuchungen und Projekte können die Statische Code Analyse daher noch hilfreicher gestalten. Die Zukunft bietet im Bereich der Code-Quality neue und weitere Herausforderungen, daher ist die Beschäftigung mit dem Thema der Statischen Code Analyse für alle Entwicklerinnen und Entwickler ein Gewinn aber auch eine Notwendigkeit.

%Your text here\ldots
%Sum up the results achieved. State current limitations of your solution. %Suggest further research by explaining how others could built on your %results.

\chapterend
