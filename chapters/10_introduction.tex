\chapter{Einleitung}

\section{Problemstellung}
Code Quality ist ein oft unterschätzter Teil der Softwareentwicklung. Viele Bugs und Sicherheitslücken können durch eine gesteigerte Code Quality vermieden werden. Um die Code Quality in einem Software-Projekt zu verbessern, werden Hilfsmittel eingesetzt, wie Tools für die Statische Code Analyse. Statische Code Analysen überprüfen den Code während der Übersetzungszeit (Compiler Übersetzung) und zeigen auftretende Fehler und verschiedene Warnungen auf. Die daraus folgenden Informationen sind aber rudimentär, d.h. text-basiert und unübersichtlich. Ebenso wird eine Verbesserung des Codestils nur schwer erreicht, da die Informationen nicht dauerhaft verfügbar sind. Die Analysen gestalten sich daher nur als Momentaufnahme aus dem Code. 

\section{Zielsetzung und Forschungsfragen}

Dadurch stellt sich die Frage, wie mit den Daten der Statischen Code Analyse weiterreichende Analysen und dauerhafte Informationsübersichten erstellt werden können und welche Möglichkeiten es gibt, daraus einen langfristigen Vorteil für die Entwicklerin oder den Entwickler zu erreichen. Das Ziel ist es daher, verschiedene Analyse- und Aufbereitungsmöglichkeiten der Daten zu entwickeln, die der Benutzerin oder den Benutzer bei der Verbesserung in der Entwicklung unterstützen und dadurch die Code Qualität in Projekten langfristig erhöht wird.

\subsection{Methodik} 



\subsection{Kriterien} 