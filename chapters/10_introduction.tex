\chapter{Einleitung}

\section{Problemstellung}
Code Quality ist ein oft unterschätzter Teil der Softwareentwicklung. Viele Bugs und Sicherheitslücken können durch eine gesteigerte Code Quality vermieden werden. Um die Code Quality in einem Software-Projekt zu verbessern, werden Hilfsmittel eingesetzt, wie Tools für die Statische Code Analyse. Statische Code Analysen überprüfen den Code während der Übersetzungszeit (Compiler Übersetzung) und zeigen auftretende Fehler und verschiedene Warnungen auf. Die daraus folgenden Informationen sind aber rudimentär, d.h. text-basiert und unübersichtlich. Ebenso wird eine Verbesserung des Codestils nur schwer erreicht, da die Informationen nicht dauerhaft verfügbar sind. Die Analysen gestalten sich daher nur als Momentaufnahme aus dem Code. 

\section{Zielsetzung und Forschungsfragen}

Dadurch stellt sich die Frage, wie mit den Daten der Statischen Code Analyse weiterreichende Analysen und dauerhafte Informationsübersichten erstellt werden können und welche Möglichkeiten es gibt, daraus einen langfristigen Vorteil für die Entwicklerin oder den Entwickler zu erreichen. Das Ziel ist es daher, verschiedene Analyse- und Aufbereitungsmöglichkeiten der Daten zu entwickeln, die der Benutzerin oder den Benutzer bei der Verbesserung in der Entwicklung unterstützen und dadurch die Code Qualität in Projekten langfristig erhöht wird.

\subsection{Methodik} 

Mithilfe einer Webapplikation werden Daten verschiedener Code-Analyse-Tools ausgewertet und präsentiert. Dazu werden in mehreren verschiedenen Projekten diese Tools eingesetzt werden. Die Analysen in der Webapplikation bauen auf diese Daten auf, die mithilfe eines Plugins und eines Datenbank-Imports gespeichert werden. Die Präsentation und Visualisierungen sollen unabhängig von den eingesetzten Tools sein.

Um die Effizienz, Anwendung und Mehrwertigkeit der Webapplikation und der Analysen festzustellen, werden Tests und Anwendungsfälle mit verschiedenen Entwicklerinnen und Entwicklern durchgeführt. Die durchführenden Benutzerinnen und Benutzer sollen hierbei einen unterschiedlichen Wissenstand im Bereich der Softwareentwicklung aufweisen, sodass die Ergebnisse nur auf der Webapplikation und nicht auf Wissen über bestimmte Tools und Fehler basieren. 


Beim Durchführen der Tests muss darauf geachtet werden, dass sich das Testsetup und die Testumgebung nicht unterscheidet. Die Ergebnisse der Tests werden protokolliert. Im Test können die Testpersonen mit der Applikation direkt und interaktiv arbeiten. Dies geschieht im Rahmen eines Interviews, wo die Anwenderinnen und Anwender Erfahrungen mit der Applikation, Kritikpunkte und Vorschläge einbringen können. Der Test beginnt mit einer zurückgesetzten Datenbank und einem leeren Frontend.ERGÄNZEN BILDSCHIRMÜBETARGUNG ODER ZIP
Der Test beinhaltet die Beantwortung von vorgefertigten Fragen, das Ausführen der Funktionen, das Ausbessern von angezeigten Fehlern sowie offenes Feedback.
Die Tests finden online statt haben einen Zeitrahmen von 30 bis 45 Minuten. 
\subsection{Kriterien} 