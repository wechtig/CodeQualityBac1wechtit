%%%%%%%%%%%%%%%%%%%%%%%%%%%%%%%%%%%%%%%%%%%%%%%%%%%%%%%%%%%%%%%%%%%%%%%%%%%%%
\chapter{Konzept}
\label{chap:concept}
%%%%%%%%%%%%%%%%%%%%%%%%%%%%%%%%%%%%%%%%%%%%%%%%%%%%%%%%%%%%%%%%%%%%%%%%%%%%%
\chapterstart
Um die beschriebenen Problemstellungen zu lösen und diese Ergebnisse auch evaluieren zu können, wird eine Applikation entwickelt. In dieser Applikation werden die Ergebnisse der Statischen Code Analyse aufbereitet, visualisiert und angezeigt. Die Daten müssen für die Möglichkeit einer langfristigen Auswertung dauerhaft in einer Datenbank gespeichert werden. 
\section{Datenbank und Testdaten}
Für die Testdaten werden gezielt Tools in verschiedenen Projekten eingesetzt. Die Daten werden in die Datenbank importiert. 
\begin{figure}[tp]
  \centering
  \includegraphics[height=6cm]{images/chartERD.PNG}
  % The short caption should be capitalised
  % The full caption should hold a full sentence. 
 \caption[Beispiel-ERD, mit der die Daten für die weitere Aufbereitung gespeichert werden können.]{Beispiel-ERD, mit der die Daten für die weitere Aufbereitung gespeichert werden können.}
  \label{fig:chartERD}
\end{figure}
Um die Daten in die Datenbank importieren zu können, ist eine zusätzliche Lösung erforderlich. Diese zusätzlich Lösung kann als Batch-Job, manueller Importer oder als Plugin erstellt werden. Da ein Batch-Job oder ein Importer eine zusätzliche Applikation erfordert, wird in Plugin implementiert. Dieses Plugin kann auch konfiguriert werden, damit die Entwicklerin oder der Entwickler die Datenquellen und den Zeitpunkt des Imports selber festlegen kann. In der Abbildung \ref{fig:chartERD} wird eine Möglichkeit dargestellt, wie diese Daten in der Datenbank gespeichert werden können und so als Basis für die Visualisierungen und Tabellen dienen können. Die Datenbank wird als nicht-relationale Datenbank erstellt, da in diesem Anwendungsfall eine nicht-relationale Lösung verschiedene Vorteile, wie die schwache Konsistenz, bietet. (siehe Punkte 4.3.1 und 4.3.3)
\section{Funktionalitäten}
In der Applikation können die Benutzerinnen und Benutzer die Daten gezielt auslesen. Da in der Datenbank eine große Menge an Daten über einen großen Zeitraum gespeichert werden sollen, ist eine Filterung der Daten notwendig. Die Visualisierungen in der Applikation betreffen die Fehler und versuchen Fragen zu lösen wie: Wo sind die meisten Fehler aufgetreten? Was sind die häufigsten Fehler und wie kann man die Fehler vermeiden? Welche Packages erfordern ein Refactoring? Diese Fragen sollen in Charts beantwortet werden, die mit den Daten automatische generiert werden. Die Applikation ist daher für den einzelnen Entwickler, als auch für das ganze Entwicklungsteam von Interesse. Diese Informationen sollen auch versendet oder gespeichert werden können, daher soll ein Export der Daten möglich sein. Der Export wir als PDF-Dokument generiert, da so alle Daten formatiert und gelistet ausgegeben werden können. Weitere Funktionalitäten wie das Ignorieren von bestimmten Meldungen oder das erstellen von Hilfen zu bestimmten Meldungen sollen den Benutzerinnen und Benutzer der Applikation unterstützen. Die Präsentation der Daten soll so erstellt werden, dass sie unabhängig der eingesetzten Tools erstellt werden können. Die Visualisierungen sollen modern und einfach verständlich präsentiert werden.
\section{Art der Applikation}
Die Applikation wird als Webapplikation entwickelt. Dies bietet den Vorteil, dass die Daten und mögliche Einstellungen und Konfigurationen für alle Entwicklerinnen und Entwickler eines Projekts oder eines Unternehmens verfügbar sind und die Webapplikation so zentral für alle verfügbar ist. In diesem Fall kann die Applikation auf einem Firmen-Server installiert werden. Ein Nachteil gegenüber einer Desktopapplikation oder einer Extension für die Entwicklungsumgebung ist hierbei die fehlende Möglichkeit, direkt und automatisch zu den Fehlerquellen zu navigieren. Der Fokus der Webapplikation richtet sich aber auf die Übersicht, Visualisierung und die kontinuierliche Verbesserung des Codes der Benutzerinnen und Benutzer. 
\section{Infrastruktur und Aufbau der Applikationen} 
Im Rahmen dieser Arbeit wurden mehrere Applikationen und Systeme verwendet. Die Arbeit setzt sich aus folgenden Teilen zusammen:

\begin{figure}[tp]
  \centering
  \includegraphics[height=7cm]{images/infrastruktur.PNG}
  % The short caption should be capitalised
  % The full caption should hold a full sentence. 
 \caption[Aufbau der Teilsysteme und Projekte der Arbeit]{Aufbau der Teilsysteme und Projekte der Arbeit.}
  \label{fig:engine}
\end{figure}

Das \textit{Project} ist das Softwareprojekt, welches analysiert werden soll. In diesem Projekt wird auch ein Tool für die Statische Code Analyse eingesetzt. Dieses Tool speichert die Daten (Errors, Warnungen, Informationen) extern. Um diese Daten für die Analyse verwenden zu können, wurde im Rahmen der Arbeit ein \textit{Plugin}, welches das Projekt einsetzten kann, entwickelt. Dieses Plugin greift auf die Daten der Statische Code Analyse zu. Um das Plugin verwenden zu können, muss auch eine \textit{Datenbank} angegeben werden, in welche das Plugin die Daten der Statischen Code Analyse des Projekts speichern kann. Die nicht-relationale Datenbank wird auf einem Server gehostet. Um in der Weboberfläche die Daten der Datenbank anzeigen zu können, wird auch ein \textit{Backend} benötigt. Das Backend liest die Daten aus und gibt sie an die Weboberfläche weiter. In der \textit{Weboberfläche} werden die Daten in verschiedenen Diagrammen und Tabellen ausgewertet. Auch ein Export der Daten als PDF-Dokument ist möglich. (siehe Punkt 4.6.2.6 Präsentation der Daten)


%Your text here\ldots
%Describe an overall concept of a solution, which could possibly solve a given problem. Design a novel solution and visualise the architecture and relevant (data) flows. Compare and relate your approach to possible alternatives and argue why the suggested solution will be better. woher testdaten; wie vorgehn bei entwicklung; was soll gelöst werden; was soll noch kommen;

\chapterend