%**********************************************************************

%---------------------------------------------------
% NOTE:
% English version of the abstract is always required 
% (even for German BA/MAs)
%---------------------------------------------------

% right side/flush
\chapterend

\begin{titlepage}

\begin{otherlanguage}{english} 
\begin{abstract} % Abstract

Code quality is an important part in the process of software development. Static code analysis is a possibility to reach a good code quality. This is the analysis of code during the compile time. The result of this analysis is the finding of bugs, code smells, security issues and other problems. The tools, that perform the static code analysis, presents the results in various forms: In desktop or web applications or in development environments. There are some different problems in these presentations, so that the developers’ code is not permanently improved. These are problems like not saving the data permanently, therefore the data can not be analyzed further. Other problems are the missing of individuality or the imperfect presentation of data.

The question is, how the information of the static code analysis should be presented, and further analysis could help the developers permanently. A web application and a plugin should solve the problem. In the web application the data is presented in different forms. The Plugin will import the data from the developers’ project into a database. 

The evolution with test subjects and the comparison with other solutions show the potential of display the information in different ways, to improve the code of the developers. The presentations include forms like visualizations, presenting tables and other possibilities to improve the code quality. 


\end{abstract}
\end{otherlanguage}


\end{titlepage}


%---------------------------------------------------
% NOTE:
% German version of the abstract "Zusammenfassung"
% is ONLY required (for German BA/MAs)
%---------------------------------------------------

\ifthenelse{\equal{\yourLanguage}{german}}{
\begin{titlepage}
\begin{otherlanguage}{german}
\begin{abstract}  % Zusammenfassung

%Zusammenfassung. (Sollte das gesamte Werk enthalten, also das spannende Problem, den gewählten --neuartigen-- Lösungsansatz und natürlich vor allem die erreichten Resultate).

Ein wichtiger Punkt in der Softwareentwicklung ist die Code Qualität. Ein Mittel, um eine hohe Code Qualität zu erreichen, ist die Statische Code Analyse. Unter der Statischen Code Analyse versteht man die Analyse des Codes zur Übersetzungszeit. Diese Analyse wird durchgeführt um Fehler, Sicherheitslücken, Code Smells und andere Probleme im Code ausfindig zu machen. Die Tools, die die Statische Code Analyse durchführen, präsentieren die Ergebnisse in verschiedenen Formen: In einer Desktopapplikation, Webapplikation oder in der Entwicklungsumgebung. Hierbei gibt es verschiedene Probleme, sodass der Code der Entwicklerinnen und Entwickler nicht dauerhaft verbessert wird. Eines der Probleme ist die fehlende dauerhafte Speicherung der Fehler. So können auch keine weiteren Analysen durchgeführt werden. Weitere Probleme sind die fehlende Individualität oder die oft rudimentär und mangelnden Anzeigen der Informationen.

Dadurch stellt sich die Frage, wie mit den Daten der Statischen Code Analyse weiterreichende Analysen und dauerhafte Informationsübersichten erstellt werden können und welche Möglichkeiten es gibt, daraus einen langfristigen Vorteil für die Entwicklerin oder den Entwickler zu erreichen. 
Dazu werden eine Webapplikation und ein Plugin entwickelt. In der Webapplikation werden die Daten der Statischen Code Analyse in verschiedenen Formen aufbereitet. Das Plugin kann von den Entwicklerinnen und Entwicklern in das Projekt eingebunden werden, sodass die Daten der Codeanalyse in die Datenbank importiert werden können.

Die Evaluierung anhand Testpersonen und Vergleiche mit herkömmlichen Lösungen zeigen, dass die Ergebnisse der Code Analyse in verschiedenen Arten angezeigt und weiter verwendet werden um den Code der Entwicklerinnen und Entwickler nachhaltig zu verbessern. Die geschieht mittels Visualisierungen, Tabellen und weiteren Möglichkeit zur Verbesserung der Code Qualität.

\end{abstract}
\end{otherlanguage}
\end{titlepage}

}{ % English Version 
% already inserted above
}

%**********************************************************************
